\documentclass[10pt, a4paper, titlepage]{article}

\usepackage{amsmath}
\usepackage{amssymb}
\usepackage{caption}
\usepackage{float}
\usepackage[portrait, margin=0.8cm]{geometry}
\usepackage{graphicx}
\graphicspath{{images/}}
\usepackage[none]{hyphenat}
\usepackage[utf8]{inputenc}
\usepackage{multicol}
\usepackage{physics}
\usepackage{sectsty}
\usepackage{subcaption}

%Removes equation label numbers
\makeatletter
\renewcommand\tagform@[1]{}
\makeatother

%Removes heading numbers of headers. Disable for numbers and to create table of contents.
\setcounter{secnumdepth}{0}

%Makes headers and footers empty: removes page numbers
\pagestyle{empty}

%Sets heading font sizes.
\sectionfont{\large}
\subsectionfont{\normalsize}

%Line between columns
%\setlength{\columnseprule}{0.4pt}

%\setlength{\parskip}{1em}

%Disables paragraph indents
\setlength{\parindent}{0em}

%Toggle column vertical alignment
\raggedcolumns

\DeclareMathOperator\cis{cis}

\title{12 Specialist Mathematics Summarised Notes \\ (Unofficial) \\ Work in Progress}
\author{}
\date{Last updated: 7 July 2022}


\begin{document}
\maketitle
\begin{multicols*}{3}

\section{Functions}
	\subsection{Composite Functions}
	Given $f:x\mapsto f(x)$ and $g:x\mapsto g(x)$, the composite function of $f$ and $g$ is:
	\begin{gather}
		(f\circ g)(x)=f(g(x))\\
		\text{or}\\
		f\circ g:x\mapsto f(g(x))
	\end{gather}
	In general, $(f\circ g)(x)\neq (g\circ f)(x)$.

	\dotfill
	\subsection{Inverse Functions}
	An inverse function returns the original value from the output of a function.\\
	$f(x)$ has an inverse if it is injective (one-to-one), if $f(a)=f(b)$ only when $a=b$, $\therefore$ passes the horizontal line test.\\\\
	For $f^{-1}(x)$, the inverse of $f(x)$:
	\begin{itemize}
		\item Is a reflection of $y=f(x)$ over $y=x$.
		\item $(f\circ f^{-1})(x)=(f^{-1}\circ f)(x)=x$
		\item Domain of $f^{-1}=$ range of $f$.
		\item Range of $f^{-1}=$ domain of $f$.
	\end{itemize}

	\dotfill
	\subsection{Self-Inverse Functions}
	An invertible function which is symmetrical about $y=x$.
	\begin{align}
		f^{-1}(x)=f(x)
	\end{align}

	\dotfill
	\subsection{Reciprocal Functions}
	A function of the form $f(x)=\cfrac{k}{x}$, where $k\neq 0$ is a constant.

	\dotfill
	\subsection{Reciprocal of Other Functions}
	The reciprocal of a function $f(x)$ is $\frac{1}{f(x)}$.\\
	\underline{Graphing $y=\frac{1}{f(x)}$ from $y=f(x)$:}
	\begin{itemize}
		\item Zero $f(x)\to$ vertical asymp $\frac{1}{f(x)}$
		\item Vertical asymp $f(x)\to$ zero $\frac{1}{f(x)}$
		\item Local max $f(x)\to$ local min $\frac{1}{f(x)}$
		\item Local min $f(x)\to$ local max $\frac{1}{f(x)}$
		\item When $f(x)>0,\ \frac{1}{f(x)}>0$
		\item When $f(x)<0,\ \frac{1}{f(x)}<0$
		\item When $f(x)\to 0,\ \frac{1}{f(x)}\to \pm \infty$
		\item When $f(x)\to \pm \infty,\ \frac{1}{f(x)}\to 0$
	\end{itemize}
	\underline{Invariant Points:}
	\\Points which do not move under a transformation occurring at $y=\pm 1$.

	\dotfill
	\subsection{Rational Functions}
	Results from the division of one polynomial by another.\\
	Vertical asymptote occurs when denominator is zero.\\
	Horizontal asymptote ascertained from behaviour of graph as $|x|\to \infty$.
	\begin{itemize}
		\item If the degree of denominator $>$ numerator, horizontal asymptote at $y=0$.
		\item If the degree of denominator $<$ numerator, function has slanted asymptote found through polynomial division.
		\item If the degree of denominator $=$ numerator horizontal asymptote at $y=\frac{a}{b}$ where $a$ and $b$ are the leading coefficients.
	\end{itemize}

	\dotfill
	\subsection{Absolute Value Functions}
	The absolute value or modulus $|x|$ of a real number $x$ is its distance from 0 on the number line.
	\begin{align}
		|x|=
		\begin{cases}
			x\,&\text{if } x\geq 0\\
			-x\,&\text{if } x < 0 
		\end{cases}
	\end{align}
	Alternatively,
	\begin{align}
		|x|=\sqrt{x^2}
	\end{align}
	\underline{Properties:}
	%\begin{align}
	%	&|x|\geq0&& &&|-x|=|x|&\\
	%	&|x|^2=x^2&& &&|xy|=|x||y|&\\
	%	&\left|\frac{x}{y}\right|=\frac{|x|}{|y|}&& &&|x-y|=|y-x|&\\
	%\end{align}
	\begin{itemize}
		\item $|x|\geq0$
		\item $|x|^2=x^2$
		\item $\left|\frac{x}{y}\right|=\frac{\left|x\right|}{\left|y\right|}$
		\item $|-x|=|x|$
		\item $|xy|=|x||y|$
		\item $|x-y|=|y-x|$
	\end{itemize}
	If $|x|=a$ where $a>0$, then $x=\pm a$.\\
	If $|x|=|b|$ then $x=\pm b$.

	\dotfill
	\subsection{Graphs Involving the Absolute Value Function}
	\underline{Graphing $y=f(|x|)$ from $y=f(x)$:}
	\begin{itemize}
		\item Discard the graph for $x<0$
		\item Reflect the graph for $x\geq 0$ in the $y$-axis
		\item Points on the $y$-axis are invariant
	\end{itemize}
	\underline{Graphing $y=|f(x)|$ from $y=f(x)$:}
	\begin{itemize}
		\item Keep the graph for $f(x)\geq 0$
		\item Reflect the graph for $f(x)<0$ in the $x$-axis
		\item Points on the $x$-axis are invariant
	\end{itemize}
	\hrulefill


\section{Trigonometric Identities}
	\subsection{Angle Relationships}
	%\begin{align}
	%	&\sin{(-\theta)}=-\sin{\theta}&& &&\cos{(-\theta)}=\cos{\theta}&\\
	%	&\sin{(\pi -\theta)}=\sin{\theta}&& &&\cos{(\pi -\theta)}=-\cos{\theta}&\\
	%	&\sin{\left(\frac{\pi}{2}-\theta \right)}=\cos{\theta}&& &&\cos{\left(\frac{\pi}{2}-\theta \right)}=\sin{\theta}&
	%\end{align}
	\resizebox{.9\linewidth}{!}{
		\begin{minipage}{\linewidth}
			\begin{align}
				&\sin{(-\theta)}=-\sin{\theta}&& &&\cos{(-\theta)}=\cos{\theta}&\\
				&\sin{(\pi -\theta)}=\sin{\theta}&& &&\cos{(\pi -\theta)}=-\cos{\theta}&\\
				&\sin{\left(\frac{\pi}{2}-\theta \right)}=\cos{\theta}&& &&\cos{\left(\frac{\pi}{2}-\theta \right)}=\sin{\theta}&
			\end{align}
		\end{minipage}
	}\\

	%\begin{flalign}
	%	&\quad \sin{(-\theta)}=-\sin{\theta}&&\\
	%	&\quad \cos{(-\theta)}=\cos{\theta}&&\\
	%	&\quad \sin{(\pi -\theta)}=\sin{\theta}&&\\
	%	&\quad \cos{(\pi -\theta)}=-\cos{\theta}&&\\
	%	&\quad \sin{\left(\frac{\pi}{2}-\theta \right)}=\cos{\theta}&&\\
	%	&\quad \cos{\left(\frac{\pi}{2}-\theta \right)}=\sin{\theta}&&
	%\end{flalign}

	\dotfill
	\subsection{Pythagorean Theorem}
	\begin{flalign}
		&\quad \sin^2{\theta}+\cos^2{\theta}=1&&\\
		&\quad \tan^2{\theta}+1=\sec^2{\theta}&&\\
		&\quad \cot^2{\theta}+1=\csc^2{\theta}&&
	\end{flalign}

	\dotfill
	\subsection{Double Angle Identities}
	\begin{flalign}
		\quad \sin{2\theta}&=2\sin{\theta}\cos{\theta}&&\\
		\quad \cos{2\theta}&=\cos^2{\theta}-\sin^2{\theta}&&\\
		&=1-2\sin^2{\theta}&&\\
		&=2\cos^2{\theta}-1&&\\
		\quad \tan{2\theta}&=\frac{2\tan{\theta}}{1-\tan^2{\theta}}&&
	\end{flalign}

	\dotfill
	\subsection{Angle Sum and Difference}
	\begin{flalign}
		&\quad \sin{(A\pm B)}=\sin{A}\cos{B}\pm \cos{A}\sin{B}&&\\
		&\quad \cos{(A\pm B)}=\cos{A}\cos{B}\mp \sin{A}\sin{B}&&\\
		&\quad \tan{(A\pm B)}=\frac{\tan{A}\pm \tan{B}}{1\mp \tan{A}\tan{B}}&&
	\end{flalign}

	\dotfill
	\subsection{Sum to Product}
	\resizebox{.85\linewidth}{!}{
		\begin{minipage}{\linewidth}
			\begin{flalign}
				&\quad \sin{A}\pm \sin{B}=2\sin{\left(\frac{A\pm B}{2}\right)}\cos{\left(\frac{A\mp B}{2}\right)}&&\\
				&\quad \cos{A}+\cos{B}=2\cos{\left(\frac{A+B}{2}\right)}\cos{\left(\frac{A-B}{2}\right)}&&\\
				&\quad \cos{A}-\cos{B}=-2\sin{\left(\frac{A+B}{2}\right)}\sin{\left(\frac{A-B}{2}\right)}&&
			\end{flalign}
		\end{minipage}
	}\\

	\dotfill
	\subsection{Product to Sum}
	\resizebox{.93\linewidth}{!}{
		\begin{minipage}{\linewidth}
			\begin{flalign}
				&\quad 2\sin{A}\cos{B}=\sin{(A+B)}+\sin{(A-B)}&&\\
				&\quad 2\sin{A}\sin{B}=\cos{(A-B)}-\cos{(A+B)}&&\\
				&\quad 2\cos{A}\cos{B}=\cos{(A+B)}+\cos{(A-B)}&&
			\end{flalign}
		\end{minipage}
	}\\

	\hrulefill


\section{Mathematical Induction}
	\subsection{The Principle of Mathematical Induction}
	Suppose $P_n$ is a proposition which is defined for every integer $n\geq a$, $a\in \mathbb{Z}$. If $P_a$ is true, and if $P_{k+1}$ is true whenever $P_k$ is true, then $P_n$ is true for all $n\geq a$.

	\hrulefill


\section{Complex Numbers}
	\subsection{Imaginary Numbers}
	A number which cannot be placed on a real number line in the form $ai$ where $a\in \mathbb{R}$ and $i=\sqrt{-1}$.\\

	\dotfill
	\subsection{Complex Numbers}
	Any number in the form $a+bi$ where $a,b\in \mathbb{R}$ and $i=\sqrt{-1}$.
	\begin{gather}
		\begin{flalign}
			&\text{If}\quad z=a+bi&&
		\end{flalign}\\
		\mathfrak{Re} (z)=a\qquad \mathfrak{Im} (z)=b
	\end{gather}

	\dotfill
	\subsection{The Complex Plane}
	Complex numbers can be plotted on the complex plane or Argand plane as a vector where the $x$-axis is the real axis and the $y$-axis is the imaginary axis.
	\begin{align}
		\overrightarrow{OP}=\begin{pmatrix}x\\ y\end{pmatrix}\quad \text{represents}\quad x+yi
	\end{align}
	\dotfill
	\subsection{Complex Conjugates}
	The complex conjugate of
	\begin{align}
		z=a+bi\qquad \text{is}\qquad z^*=a-bi
	\end{align}
	In the complex plane, $z^*$ is the reflection of $z$ in the real axis.

	\dotfill
	\subsection{Modulus and Argument}
	The modulus of the complex number \\$z=a+bi$ is the length of the vector $\big(\begin{smallmatrix}a\\ b\end{smallmatrix}\big)$, which is the real number:
	\begin{align}
		|z|=\sqrt{a^2+b^2}
	\end{align}
	The argument of $z$, $\arg({z})$ is the angle $\theta$ between the positive real axis and $\big(\begin{smallmatrix}a\\ b\end{smallmatrix}\big)$.\\
	Real numbers have an argument of 0 or $\pi$.\\
	Purely imaginary numbers have argument of $\frac{\pi}{2}$ or $-\frac{\pi}{2}$.\\

	\underline{Properties of Modulus:}
	\begin{itemize}
		\item $|z^*|=|z|$
		\item $|z^*|^2=zz^*$
		\item $|z_1z_2|=|z_1||z_2|$
		\item $\left|\frac{z_1}{z_2}\right|=\frac{|z_1|}{|z_2|},\ z_2\neq 0$
		\item $|z_1z_2z_3\dots z_n|=|z_1||z_2||z_3|\dots |z_n|$
		\item $|z^n|=|z|^n,\ n\in \mathbb{Z}^+$
	\end{itemize}

	\dotfill
	\subsection{Polar Form}
	%\begin{gather}
	%	\cis{\theta}=\cos{\theta}+i\sin{\theta}\\
	%	\begin{flalign}
	%		&\text{A complex number $z$ has polar form}&&
	%	\end{flalign}\\
	%	z=|z|\cis{\theta}\\
	%	\begin{flalign}
	%		&\text{where $\theta = \arg{z}$.}&&\\
	%		&\text{The conjugate of $z$ is:}&&
	%	\end{flalign}\\
	%	z^*=|z|\cis{-\theta}
	%\end{gather}\\
	\begin{align}
		\cis{\theta}=\cos{\theta}+i\sin{\theta}
	\end{align}
	A complex number $z$ has polar form
	\begin{align}
		z=|z|\cis{\theta}
	\end{align}
	where $\theta = \arg({z})$.\\
	The conjugate of $z$ is:
	\begin{align}
		z^*=|z|\cis{(-\theta)}
	\end{align}
	\underline{Properties of $\cis{\theta}$:}
	\begin{itemize}
		\item $\cis{\theta}\times \cis{\phi}=\cis{(\theta +\phi )}$
		\item $\frac{\cis{\theta}}{\cis{\phi}}=\cis{(\theta -\phi )}$
		\item $\cis{(\theta -2k\pi )}=\cis{\theta},\ k\in \mathbb{Z}$
	\end{itemize}

	% NOTE: Not in the course
	%\dotfill
	%\subsection{Euler's Form}
	%\begin{align}
	%	e^{i\theta}=\cos{\theta}+i\sin{\theta}
	%\end{align}

	\dotfill
	\subsection{De Moivre's Theorem}
	\begin{align}
		(|z|\cis{\theta})^n=|z|^n\cis{n\theta},\ \text{for all }n\in \mathbb{Q}
	\end{align}
	\dotfill
	\subsection{Roots of Complex Numbers}
	The $n^{\text{th}}$ roots of the complex number $c$ are the solutions of $z^n=c$.

	\dotfill
	\subsection{The $n^{\text{th}}$ Roots of Unity}
	The $n^{\text{th}}$ roots of unity are the solutions of $z^n = 1$.
	
	\dotfill
	\subsection{Distances in the Complex Plane}
	If $z_1\equiv \overrightarrow{OP_1}$ and $z_2\equiv \overrightarrow{OP_2}$ then $|z_1-z_2|$ is the distance between points $P_1$ and $P_2$.

	\hrulefill

\section{Real Polynomials}
	\subsection{Zeros and Roots}
	A zero of a polynomial is a value of the variable which makes the polynomial equal to zero.\\
	$\alpha$ is a zero of polynomial
	\begin{align}
		P(x)\iff P(\alpha)=0
	\end{align}
	The roots of a polynomial equation are the solutions to the equation.\\
	$\alpha$ is a root (or solution) of
	\begin{align}
		P(x)\iff P(\alpha)=0
	\end{align}
	The roots of $P(x)=0$ are the zeros of $P(x)$ and the $x$-intercepts of the graph $y=P(x)$
	
	\dotfill
	\subsection{Factors}
	$(x-\alpha)$ is a factor of the polynomial $P(x)\iff$ there exists a polynomial $Q(x)$ such that $P(x)=(x-\alpha)Q(x)$.
	
	\dotfill
	\subsection{Polynomial Equality}
	Two polynomials are equal if and only if they have the same degree (order) and corresponding terms have equal coefficients.
	
	\dotfill
	\subsection{Polynomial Division by Linears}
	If $P(x)$ is divided by $D(x)=ax+b$ until a quotient $Q(x)$ and constant remainder $R$ is obtained, then
	\begin{align}
		\frac{P(x)}{ax+b}=Q(x)+\frac{R}{ax+b}
	\end{align}
	Notice that $P(x)=Q(x)\times (ax+b)+R$.
	
	\dotfill
	\subsection{Polynomial Division by Quadratics}
	If $P(x)$ is divided by $D(x)=ax^2+bx+c$, then
	\begin{align}
		\frac{P(x)}{ax^2+bx+c}=Q(x)+\frac{ex+f}{ax^2+bx+c}
	\end{align}
	where $ex+f$ is the remainder.
	
	\dotfill
	\subsection{The Remainder Theorem}
	When a polynomial $P(x)$ is divided by $x-k$ until a constant remainder $R$ is obtained, then $R=P(k)$.
	
	\dotfill
	\subsection{The Factor Theorem}
	For any polynomial $P(x)$, $k$ is a zero of $P(x)\iff (x-k)$ is a factor of $P(x)$.
	
	\dotfill
	\subsection{The Fundamental Theorem of Algebra}
	If $P(x)$ is a polynomial of degree $n$, then $P(x)$ has $n$ zeros, each in the form $a+bi$ where $a,b\in \mathbb{R}$, some of which may be repeated.
	
	% TODO: Things not in the course can be highlighted in red to indicate extra content
	% NOTE: Not in the course

	%\dotfill
	%\subsection{Sum and Product of Roots}
	%For the polynomial equation
	%\begin{gather}
	%	\sum_{r=0}^{n}a_rx^r=0,\quad a_n\neq 0\\
	%	\begin{align}
	%		\text{Sum of roots: }&\frac{-a_{n-1}}{a_n}\\
	%		\text{Product of roots: }&\frac{(-1)^na_0}{a_n}
	%	\end{align}	
	%\end{gather}
	\hrulefill

\section{Vectors}
	\subsection{Vectors in Space}
	Any point $P$ in space can be specified $(x,y,z)$ corresponding to steps in the $X$, $Y$ and $Z$ direction from the origin $O$.\\
	The position vector of $P$ is
	\begin{align}
		\overrightarrow{OP}=\begin{pmatrix}x\\ y \\z\end{pmatrix}=x\vb{i}+y\vb{j}+z\vb{k}
	\end{align}
	where $\vb{i}=\big(\begin{smallmatrix}1 \\0 \\0 \end{smallmatrix}\big)$, 
	$\vb{j}=\big(\begin{smallmatrix}0 \\1 \\0 \end{smallmatrix}\big)$, and 
	$\vb{k}=\big(\begin{smallmatrix}0 \\0 \\1 \end{smallmatrix}\big)$, the base unit vectors.

	\dotfill
	\subsection{The Magnitude of a Vector}
	The magnitude or length of the vector $\vb{a}=\big(\begin{smallmatrix}a_1 \\a_2 \\a_3 \end{smallmatrix}\big)$ is
	\begin{align}
		|\vb{a}|=\sqrt{a_1^2+a_2^2+a_3^2}
	\end{align}
	
	\dotfill
	\subsection{Operations with Vectors}
	\begin{gather}
	\begin{align}
		\text{If }\vb{a}=
		\begin{pmatrix}a_1\\ a_2 \\a_3\end{pmatrix}
		\text{ and }\vb{b}=
		\begin{pmatrix}b_1\\ b_2 \\b_3\end{pmatrix}
		\text{ then:}
	\end{align}\\
	\resizebox{.85\linewidth}{!}{
		\begin{minipage}{\linewidth}
			\begin{align}
				&-\vb{a}=\begin{pmatrix}-a_1\\ -a_2 \\-a_3\end{pmatrix}&& &&\vb{a}+\vb{b}=\begin{pmatrix}a_1+b_1\\ a_2+b_2 \\a_3+b_3\end{pmatrix}&\\
				&\vb{a}-\vb{b}=\begin{pmatrix}a_1-b_1\\ a_2-b_2 \\a_3-b_3\end{pmatrix}&& &&k\vb{a}=\begin{pmatrix}ka_1\\ ka_2 \\ka_3\end{pmatrix}&
			\end{align}
		\end{minipage}
	}
	\end{gather}

	\dotfill
	\subsection{Vector Algebra}
	\begin{itemize}
		\item $\vb{a}+\vb{b}=\vb{b}+\vb{a}$
		\item $(\vb{a}+\vb{b})+\vb{c}=\vb{a}+(\vb{b}+\vb{c})$
		\item $\vb{a}+\vb{0}=\vb{0}+\vb{a}$
		\item $\vb{a}+(\vb{-a})=(\vb{-a})+\vb{a}=\vb{0}$
		\item $k(\vb{a}+\vb{b})=k\vb{a}+k\vb{b}$
		\item $|k\vb{a}|=|k||\vb{a}|$
		\item If $\vb{c}+\vb{a}=\vb{b}$ then $\vb{c}=\vb{b}-\vb{a}$
		\item If $k\vb{b}=a,$ $k\neq 0$, then $\vb{b}=\frac{1}{k}\vb{a}$
	\end{itemize}

	\dotfill
	\subsection{Vector Between Two Points}
	If $A(a_1,a_2,a_3)$ and $B(b_1,b_2,b_3)$ then the position vector of $B$ relative to $A$ is 
	\begin{align}
		\overrightarrow{AB}=\overrightarrow{OB}-\overrightarrow{OA}=\begin{pmatrix}b_1-a_1\\ b_2-a_2 \\b_3-a_3\end{pmatrix}
	\end{align}
	The distance from $A$ to $B$ is
	\resizebox{.85\linewidth}{!}{
		\begin{minipage}{\linewidth}
			\begin{align}
				|\overrightarrow{AB}|=\sqrt{(b_1-a_1)^2+(b_2-a_2)^2+(b_3-a_3)^2}
			\end{align}
		\end{minipage}
	}\\
	
	\dotfill
	\subsection{Unit Vector}
	For a vector $\vb{u}$, the unit vector would be:
	\begin{equation}
		\hat{\vb{u}} = \frac{\vb{u}}{|\vb{u}|}
	\end{equation}

	\dotfill
	\subsection{Dot Product (Scalar Product)}
	The algebraic definition of the dot product is defined as thus:
	\begin{equation}
		\vb{a} \cdot \vb{b} = \sum_{i=1}^{n} a_i b_i
	\end{equation}
	The geometric definition is as follows:
	\begin{equation}
		\vb{a} \cdot \vb{b} = |\vb{a}||\vb{b}|\cos\theta
	\end{equation}

	\dotfill
	\subsection{Properties of the Dot Product}
	\begin{itemize}
		\item $\vb{a}\cdot\vb{b} = \vb{b}\cdot\vb{a}$
		\item $\vb{a}\cdot\vb{a} = |\vb{a}|^2$
		\item $\vb{a}\cdot(\vb{b}+\vb{c})=\vb{a}\cdot\vb{b}+\vb{a}\cdot\vb{c}$
		\item $\lambda(\vb{a}\cdot\vb{b})=(\lambda\vb{a})\cdot\vb{b}=\vb{a}\cdot(\lambda\vb{b}), \,\lambda\in\mathbb{R}$
	\end{itemize}

	\hrulefill


%\pagebreak
%\section{Integration}
%	\subsection{Integration by Parts}
%	Suppose we have a function $f = u\times v$ where $u$ and $v$ are also functions. We know that to find the derivative of $f$, we utilise the product rule:
%	\begin{align}
%		\dv{f}{x} = \dv{(u \times v)}{x} = u\dv{v}{x} + v\dv{u}{x}
%	\end{align}
%	Then, integrate both sides with respect to $x$:
%	\begin{align}
%		\int{\dv{(u \times v)}{x} \, \dd{x}} = \int{u\dv{v}{x} \, \dd{x}} + \int{v\dv{u}{x} \, \dd{x}}
%	\end{align}
%	Simplifying, we get:
%	\begin{align}
%		uv = \int{u\dv{v}{x} \, \dd{x}} + \int{v\dv{u}{x} \, \dd{x}}
%	\end{align}
%	Rearranging:
%	\begin{align}
%		\int{u\dv{v}{x} \, \dd{x}} = uv - \int{v\dv{u}{x} \, \dd{x}}
%	\end{align}


\end{multicols*}
\end{document}
