\documentclass[10pt, a4paper, titlepage]{article}

% ---
% --- Formatting
\usepackage[portrait, margin=0.8cm]{geometry}
\usepackage{caption}
\usepackage{float}
\usepackage[none]{hyphenat}
\usepackage{multicol}
\usepackage{sectsty}
\usepackage{subcaption}

% Removes heading numbers of headers. Disable for numbers and to create a table of contents.
\setcounter{secnumdepth}{0}

% Makes headers and footers empty: removes page numbers
\pagestyle{empty}

% Line between columns
%\setlength{\columnseprule}{0.4pt}

%\setlength{\parskip}{1em}

% Disables paragraph indents
\setlength{\parindent}{0em}

% Toggle column vertical alignment
\raggedcolumns


% ---
% --- Typography
\usepackage[utf8]{inputenc}

% Sets heading font sizes.
\sectionfont{\large}
\subsectionfont{\normalsize}

% ---
% --- Graphics
\usepackage{graphicx}
\graphicspath{{images/}}


% ---
% --- Math
\usepackage{amsmath}
\usepackage{amssymb}
\usepackage{physics}

% Removes equation label numbers
\makeatletter
\renewcommand\tagform@[1]{}
\makeatother



\begin{document}

\title{11 Specialist Mathematics Semester 2 Summarised Notes}
\author{}
\date{2021}
\maketitle

% Begin content
\begin{multicols*}{3}
\section{Matrices}
\subsection{Notation}
\begin{align}
	\begin{bmatrix}
		a & b & c\\
		d & e & f
	\end{bmatrix}
\end{align}
Order: $2\times 3$

\dotfill
\subsection{Addition and Subtraction}
\begin{align}
	\begin{bmatrix}
		a & b\\
		c & d
	\end{bmatrix}
	\pm
	\begin{bmatrix}
		e & f\\
		g & h
	\end{bmatrix}
	=
	\begin{bmatrix}
		a\pm e & b\pm f\\
		c\pm g & d\pm h
	\end{bmatrix}
\end{align}
\dotfill
\subsection{Scalar Multiplication}
\begin{align}
	k\times
	\begin{bmatrix}
		a & b & c\\
		d & e & f
	\end{bmatrix}
	=
	\begin{bmatrix}
		ka & kb & kc\\
		kd & ke & kf
	\end{bmatrix}
\end{align}
\dotfill
\subsection{Matrix Multiplication}
\begin{align}
	\begin{bmatrix}
		a & b\\
		c & d
	\end{bmatrix}
	\begin{bmatrix}
		w & x\\
		y & z
	\end{bmatrix}
	=
	\begin{bmatrix}
		aw+by & ax+bz\\
		cw+dy & cx+dz
	\end{bmatrix}
\end{align}
Order:\\
$n\times m\ \leftarrow$ must match $\rightarrow \ m\times p$\\
Resultant matrix: $n\times p$\\

\dotfill
\subsection{Identity Matrix}
\begin{align}
	I=
	\begin{bmatrix}
		1 & 0\\
		0 & 1
	\end{bmatrix}
	\qquad AI=IA=A
\end{align}

\dotfill
\subsection{Inverse of a Matrix}
\begin{gather}
	A^{-1}=\frac{1}{ad-bc}
	\begin{bmatrix}
		d & -b\\
		-c & a
	\end{bmatrix}\\
	\det{A}=|A|=ad-bc
\end{gather}
\begin{itemize}
	\item If $\det{A}\neq 0$, $A$ is invertible
	\item If $\det{A}=0$, $A$ is singular
\end{itemize}
\begin{align}
	AA^{-1}=A^{-1}A=I
\end{align}
\dotfill
\subsection{Matrix Algebra}
\begin{itemize}
	\item $A+O=O+A=A$ where $O$ is zero/null matrix
	\item $A+(-A)=(-A)+A=O$
	\item $A+B$ exists if both have same order
	\item $A+B=B+A$ \{commutative\}
	\item $(A+B)+C=A+(B+C)$ \{associative\}
	\item In general, $AB\neq BA$ \{non-commutative\}
	\item $(AB)C=A(BC)$ \{associative\}
	\item $AO=OA=O$
	\item $AB$ may be $O$ without $A=O$ or $B=O$
	\item $A(B+C)=AB+AC$ \{distributive law\}
	\item $AI=IA=A$
	\item $A^n$ exists provided $A$ is square and $n\in \mathbb{Z}^+$
	\item $AA^{-1}=A^{-1}A=I$
	\item $(A^{-1})^{-1}=A$
	\item $(kA)^{-1}=\frac{1}{k}A^{-1}$
	\item $(AB)^{-1}=B^{-1}A^{-1}$
\end{itemize}
\dotfill
\subsection{Simultaneous Linear Equations}
\begin{gather}
	\begin{cases}
		2x+3y=4\\
		5x+4y=17
	\end{cases}\\
	\begin{bmatrix}
		2 & 3\\
		5 & 4
	\end{bmatrix}
	\begin{bmatrix}
		x\\
		y
	\end{bmatrix}
	=
	\begin{bmatrix}
		4\\
		17
	\end{bmatrix}\\
	AX=B,\ X=A^{-1}B\\
	\begin{align}
		\begin{bmatrix}
			x\\
			y
		\end{bmatrix}
		&=-\frac{1}{7}
		\begin{bmatrix}
			4 & -3\\
			-5 & 2
		\end{bmatrix}
		\begin{bmatrix}
			4\\
			17
		\end{bmatrix}\\
		\begin{bmatrix}
			x\\
			y
		\end{bmatrix}
		&=
		\begin{bmatrix}
			5\\
			-2
		\end{bmatrix}
	\end{align}\\
	\therefore x=5,\ y=-2
\end{gather}
\dotfill
\subsection{Translations and Lines in 2D}
\begin{gather}
	A=\begin{pmatrix}a_1\\ a_2\end{pmatrix}
	\text{ translated through }
	b=\begin{pmatrix}b_1\\ b_2\end{pmatrix}\\
	A'=\begin{pmatrix}a_1+b_1\\ a_2+b_2\end{pmatrix}
\end{gather}
Line:
\begin{align}
	\begin{pmatrix}x\\ y\end{pmatrix}=\begin{pmatrix}a_1\\ a_2\end{pmatrix}+\lambda \begin{pmatrix}b_1\\ b_2\end{pmatrix}
\end{align}
Where $\lambda$ is a parameter\\

\dotfill
\subsection{Linear Transformations}
Maps an object initial vector onto its image.\\
$T$ is a linear transformation if:
\begin{itemize}
	\item $T(\vec{u}+\vec{v})=T(\vec{u})+T(\vec{v})$
	\item $T(k\vec{u})=kT(\vec{u})$
	\item $T(\vec{0})=\vec{0}$
	\item $T(-\vec{u})=-T(\vec{u})$
	\item $T(k_1\vec{u_1}+k_2\vec{u_2}+\dots +k_r\vec{u_r})\\ =k_1T(\vec{u_1})+k_2T(\vec{u_2})+\dots +k_rT(\vec{u_r})$
\end{itemize}
\dotfill
\subsection{Geometric Linear Transformations}
\begin{gather}
	\begin{bmatrix}
		x'\\
		y'
	\end{bmatrix}
	=
	\begin{bmatrix}
		a & b\\
		c & d
	\end{bmatrix}
	\begin{bmatrix}
		x\\
		y
	\end{bmatrix}\\
	v'=Av,\ v=A^{-1}v'
\end{gather}
To transform a function or relation, substitute in $x$ and $y$ from $v$.\\

\dotfill
\subsection{Rotations About the Origin}
For a rotation anticlockwise about $O(0,0)$ through $\theta$,
\begin{align}
	A=
	\begin{bmatrix}
		\cos{\theta} & -\sin{\theta}\\
		\sin{\theta} & \cos{\theta}
	\end{bmatrix}
\end{align}
with $\det{A}=1$\\

\dotfill
\subsection{Reflections}
For a reflection in the mirror line \\$y=(\tan{\alpha})x$,
\begin{align}
	A=
	\begin{bmatrix}
		\cos{2\alpha} & \sin{2\alpha}\\
		\sin{2\alpha} & -\cos{2\alpha}
	\end{bmatrix}
\end{align}
with $\det{A}=-1$\\
If $m=\tan{\alpha}$ then:
\begin{align}
	&\cos{2\alpha}=\frac{1-m^2}{1+m^2}&& &&\sin{2\alpha}=\frac{2m}{1+m^2}&\\
	&\tan{2\alpha}=\frac{2m}{1-m^2}&&	
\end{align}
\dotfill
\subsection{Dilatations}
For a dilation with scale factor $m$ parallel to the $x$-axis and $k$ parallel to the $y$-axis:
\begin{align}
	A=
	\begin{bmatrix}
		m & 0\\
		0 & k
	\end{bmatrix}
\end{align}
\dotfill
\subsection{Compositions of Transformations}
If $v$ is transformed under $T_A$ followed by $T_B$,
\begin{align}
	v'=BAv
\end{align}
The transformation is $T_B\circ T_A$\\

\hrulefill
\section{Real and Complex Numbers}
\subsection{Number Sets}
\begin{flalign}
	&\mathbb{N}=\{1,2,3,4,\dots\}\text{ natural numbers}&&\\
	&\mathbb{Z}=\{0,\pm 1,\pm 2,\pm 3,\dots\}\text{ all integers}&&\\
	&\mathbb{Z}^+=\{1,2,3,\dots\}\text{ positive integers}&&\\
	&\mathbb{Z}^-=\{-1,-2,-3,\dots\}\text{ negative integers}&&\\
	&\mathbb{Q}\text{ all rational numbers}&&\\
	&\mathbb{Q}'\text{ all irrational numbers}&&\\
	&\mathbb{R}\text{ all real numbers}&&\\
	&\mathbb{I}\text{ all imaginary numbers}&&\\
	&\mathbb{C}\text{ all complex numbers}&&
\end{flalign}
\dotfill
\subsection{Real Numbers}
Numbers that can be placed on a number line.\\

\dotfill
\subsection{Rational Numbers}
A real number which can be written in the form $\frac{p}{q}$ where $p,q\in \mathbb{Z}$ and $q\neq 0$. Have a decimal expansion that either terminates or recurs.\\

\dotfill
\subsection{Irrational Numbers}
A real number which cannot be written in the form $\frac{p}{q}$ where $p,q\in \mathbb{Z}$ and $q\neq 0$. For example, surds, $(\sqrt{a})$.\\

\dotfill
\subsection{Interval Notation}
\begin{align}
	A=\{x\ |\ a\leq x\leq b,\ x\in \mathbb{R}\}
\end{align}
``The set of real numbers $x$ such that $x$ is between $a$ and $b$, including $a$ and $b$."
\begin{itemize}
	\item An interval is a connected subset of the number line $\mathbb{R}$.
	\item An interval is closed if both of its endpoints are included.
	\item An interval is open if both of its endpoints are not included.
\end{itemize}
\begin{align}
	[a,b]\quad \text{is}\quad \{x\ |\ a\leq x\leq b,\ x\in \mathbb{R}\}\\
	[a,b)\quad \text{is}\quad \{x\ |\ a\leq x<b,\ x\in \mathbb{R}\}\\
	(a,b]\quad \text{is}\quad \{x\ |\ a<x\leq b,\ x\in \mathbb{R}\}\\
	(a,b)\quad \text{is}\quad \{x\ |\ a<x<b,\ x\in \mathbb{R}\}\\
\end{align}
\dotfill
\subsection{Set Notation}
\begin{align}
	&\in \quad \text{element of}&& &&\notin \quad \text{not element}&\\
	&\subset \quad \text{subset}&& &&\not\subset \quad \text{not subset}&\\
	&\subseteq \quad \text{equal subset}&& &&\varnothing \quad \text{empty set}\\
	&\cup \quad \text{union}&& &&\cap \quad \text{intersection}&	
\end{align}
\dotfill
\subsection{Imaginary Numbers}
A number which cannot be placed on a number line in the form $ai$ where $a\in \mathbb{R}$ and $i=\sqrt{-1}$.\\

\dotfill
\subsection{Complex Numbers}
Any number in the form $a+bi$ where $a,b\in \mathbb{R}$ and $i=\sqrt{-1}$.
\begin{gather}
	\begin{flalign}
		&\text{If}\quad z=a+bi&&
	\end{flalign}\\
	\mathfrak{Re} (z)=a\qquad \mathfrak{Im} (z)=b
\end{gather}
\dotfill
\subsection{Conjugates (Surds)}
Rationalising the denominator.
\begin{align}
	\frac{a+\sqrt{b}}{c+\sqrt{d}}=\frac{a+\sqrt{b}}{c+\sqrt{d}}\times \frac{c-\sqrt{d}}{c-\sqrt{d}}
\end{align}
\dotfill
\subsection{Complex Conjugates}
\begin{align}
	z=a+bi\qquad \text{and}\qquad z^*=a-bi
\end{align}
are complex conjugates. To write a fraction of two complex numbers with a real denominator:
\begin{gather}
	\frac{z}{w}=\frac{z}{w}\times \frac{w^*}{w^*}=\frac{zw^*}{ww^*}\\
	\frac{a+bi}{c+di}=\frac{a+bi}{c+di}\times \frac{c-di}{c-di}
\end{gather}
Properties:
\begin{itemize}
	\item $(z^*)^*=z$
	\item $(z_1+z_2)^*=z_1^*+z_2^*$
	\item $(z_1z_2)^*=z_1^*\times z_2^*$
	\item $\left(\frac{z_1}{z_2}\right)^*=\frac{z_1^*}{z_2^*}$
	\item $(z^n)^*=(z^*)^n,\ n\in \mathbb{Z}^+$
	\item $z+z^*$ and $zz^*$ are real
\end{itemize}
\dotfill
\subsection{Complex Solutions of Quadratics}
If a real quadratic equation has $\Delta <0$, and root $c+di$, the other root is its conjugate $c-di$.\\
Then the equation is:
\begin{align}
	a\left(x^2-2cx+\left(c^2+d^2\right)\right)=0,\ a\neq 0
\end{align}
For the roots of $ax^2+bx+c=0$:
\begin{align}
	\text{sum}=-\frac{b}{a}\qquad \text{product}=\frac{c}{a}
\end{align}
\dotfill
\subsection{Complex Numbers as 2D Vectors}
Complex numbers can be represented on an Argand diagram where the $x$-axis is the real axis and the $y$-axis is the imaginary axis.
\begin{align}
	\overrightarrow{OP}=\begin{pmatrix}x\\ y\end{pmatrix}\quad \text{represents}\quad x+yi
\end{align}
\dotfill
\subsection{Modulus}
The modulus of the complex number \\$z=a+bi$ is the length of the vector $\big(\begin{smallmatrix}a\\ b\end{smallmatrix}\big)$, which is the real number:
\begin{align}
	|z|=\sqrt{a^2+b^2}
\end{align}
Properties:
\begin{itemize}
	\item $|z^*|=|z|$
	\item $|z^*|=zz^*$
	\item $|z_1z_2|=|z_1||z_2|$
	\item $\left|\frac{z_1}{z_2}\right|^*=\frac{|z_1|}{|z_2|},\ z_2\neq 0$
	\item $|z_1z_2z_3\dots z_n|=|z_1||z_2||z_3|\dots |z_n|$
	\item $|z^n|=|z|^n,\ n\in \mathbb{Z}^+$
\end{itemize}
\dotfill
\subsection{Coordinate Geometry}
\begin{align}
	\text{If}\quad z_1\equiv \overrightarrow{OP_1}\quad \text{and}\quad z_2\equiv \overrightarrow{OP_2}
\end{align}
Distance between $P_1$ and $P_2$ is:
\begin{align}
	|z_1-z_2|
\end{align}
If $M$ is the midpoint between $P_1$ and $P_2$,
\begin{align}
	\overrightarrow{OM}\equiv \frac{z_1+z_2}{2}
\end{align}
\hrulefill
\pagebreak
\section{Sequences and Series}
\subsection{Arithmetic Sequences}
Sequence where each term differs from the previous by a fixed term, the common difference.\\
First term: $t_1$\\
Common difference: $d$\\
General Formula:
\begin{align}
	t_n=t_1+(n-1)d
\end{align}
\dotfill
\subsection{Geometric Sequences}
Sequence where each is the product of the previous term and a fixed common ratio.\\
First term: $t_1$\\
Common ratio: $r$\\
General Formula:
\begin{align}
	t_n=t_1r^{n-1}
\end{align}
\dotfill
\subsection{Series}
A series is the sum of the terms in a sequence.\\
\underline{Sigma Notation:}
\begin{align}
	\sum_{k=1}^nt_k=t_1+t_2+t_3+\dots +t_n
\end{align}
Properties:
\begin{flalign}
	&\quad \sum_{k=1}^n(a_k+b_k)=\sum_{k=1}^na_k+\sum_{k=1}^nb_k&&\\
	&\text{If }c \text{ is a constant,}&&\\
	&\quad \sum_{k=1}^nca_k=c\sum_{k=1}^na_k&&\\
	&\quad \sum_{k=1}^nc=cn&&
\end{flalign}
\dotfill
\subsection{Arithmetic Series}
The sum of the terms of an arithmetic sequence.
\begin{gather}
	S_n=\sum_{k=1}^n\left(t_1+(k-1)d\right)\\
	S_n=\frac{n}{2}(t_1+t_n)\\
	\text{or}\\
	S_n=\frac{n}{2}(2t_1+(n-1)d)
\end{gather}
\dotfill
\subsection{Geometric Series}
The sum of the terms of a geometric sequence.
\begin{gather}
	S_n=\sum_{k=1}^nt_1r^{k-1}\\
	S_n=\frac{t_1(r^n-1)}{r-1}\\
	\text{or}\\
	S_n=\frac{t_1(1-r^n)}{1-r}\\
	\begin{flalign}
		&\qquad r\neq 1&&
	\end{flalign}
\end{gather}
\underline{Sum of Infinite Geometric Series:}\\
If $|r|<1$, an infinite series of the form
\begin{align}
	t_1+t_1r+t_1r^2+\dots =\sum_{k=1}^\infty t_1r^{k-1}
\end{align}
will converge to the sum
\begin{align}
	S=\frac{t_1}{1-r}
\end{align}
\hrulefill
\end{multicols*}
\end{document}
